\documentclass[
	12pt,					% tamanho da fonte
	openright,				% capítulos começam em pág ímpar (insere página vazia caso preciso)
	oneside,				% para impressão em recto e verso (twoside). Oposto a (oneside)
	a4paper,				% tamanho do papel. 
	chapter=TITLE,			% títulos de capítulos convertidos em letras maiúsculas
	section=TITLE,			% títulos de seções convertidos em letras maiúsculas
	sumario=abnt-6027-2012,
	english,				% idioma adicional para hifenização
	brazil,					% o último idioma é o principal do documento
	fleqn,					% equações alinhadas a esquerda (UDESC/CCT)+
	]{abntex2}

% ----------------------------------------------------------
% Pacotes básicos 
% ----------------------------------------------------------
\usepackage{amsmath}							% Pacote matemático
\usepackage{amssymb}							% Pacote matemático
\usepackage{amsfonts}							% Pacote matemático
%\usepackage{lmodern}							% Usa a fonte Latin Modern		
\usepackage{mathptmx} 							% Usa a fonte Times New Roman
\usepackage[T1]{fontenc}						% Selecao de codigos de fonte.
\usepackage[utf8]{inputenc}						% Codificacao do documento (conversão automática dos acentos)
\usepackage{lastpage}							% Usado pela Ficha catalográfica
\usepackage{indentfirst}						% Indenta o primeiro parágrafo de cada seção.
\usepackage[dvipsnames,table]{xcolor}			% Controle das cores
\usepackage{graphicx}							% Inclusão de gráficos
\usepackage{microtype} 							% para melhorias de justificação
\usepackage{lipsum}								% para geração de dummy text
\usepackage[brazilian,hyperpageref]{backref}	% Paginas com as citações na bibl
\usepackage[alf,abnt-emphasize=bf,abnt-full-initials=yes]{abntex2cite}					% Citações padrão ABNT
%\usepackage[num]{abntex2cite}					% Citações padrão ABNT numérica
\usepackage{adjustbox}							% Pacote de ajuste de boxes
\usepackage{subcaption}							% Inclusão de Subfiguras e sublegendas		
\usepackage{enumitem}							% Personalização de listas
\usepackage{siunitx}							% Grandezas e unidades
\usepackage[section]{placeins}					% Manter as figuras delimitadas na respectiva seção com a opção [section]
\usepackage{multirow}							% Multi colunas nas tabelas
\usepackage{array,tabularx} 					% Pacotes de tabelas
\usepackage{booktabs}							% Pacote de tabela profissonal
\usepackage{rotating}							% Rotacionar figuras e tabelas
\usepackage{xfrac}								% Fazer frações n/d em linha
\usepackage{bm}									% Negrito em modo matemático
\usepackage{xstring}							% Manipulação de strings
\usepackage{pgfplots}							% Pacote de Gráficos
\usepackage{tikz}								% Pacote de Figuras
\usepackage[american, cuteinductors,smartlabels, fulldiode, siunitx, americanvoltages, oldvoltagedirection, smartlabels]{circuitikz}						% Pacote de circuitos elétricos
\usepackage{chemformula}						% Pacote para fórmulas químicas
\usepackage{chngcntr}							% Pacte usado para deixar numeração de equações sequencial (UDESC/CCT)
\counterwithout{equation}{chapter}
% fonte: https://latex.org/forum/viewtopic.php?t=15392

% Comando para deixar numeração das equações contínua (1), (2), (3)... ao invés de organizar por capítulos (1.1)(1.2)... (2.1)(2.2)
%\renewcommand{\theequation}{\arabic{equation}}

%\numberwithin{equation}{section}


% Cabecalho cabeçalho somente com numeração de página 10pt
\makepagestyle{PagNumReduzida}
\makeevenhead{PagNumReduzida}{\ABNTEXfontereduzida\thepage}{}{}
\makeoddhead{PagNumReduzida}{}{}{\ABNTEXfontereduzida\thepage}
%fonte: https://github.com/abntex/abntex2/wiki/HowToCustomizarCabecalhoRodape
%fonte: Manual memoir seção 7.3 pg. 111 pdf http://linorg.usp.br/CTAN/macros/latex/contrib/memoir/memman.pdf 

% Personalização das opções das listas
\setlist[itemize]{leftmargin=\parindent}

% Citação online --- MODIFICAR ---
\newcommand{\citeshort}[1]{\citeauthoronline{#1}~(\citeyear{#1})}

\newcommand{\me}[1]{Elaborado pelo autor (#1).}

% Configuração do pgfplots
\pgfplotsset{compat=newest} %compat=1.14
\pgfplotsset{plot coordinates/math parser=false} 
\newlength\figureheight 
\newlength\figurewidth 

% Libraries do TiKz
\usetikzlibrary{quotes,angles,arrows}
\usetikzlibrary{through,calc,math}
\usetikzlibrary{graphs,backgrounds,fit}
\usetikzlibrary{shapes,positioning,patterns,shadows}
\usetikzlibrary{decorations.pathreplacing}
\usetikzlibrary{shapes.geometric}
\usetikzlibrary{arrows.meta}
\usetikzlibrary{external}

%\tikzexternalize[]
%\tikzexternalenable
%\tikzexternalize
%\tikzexternaldisable
%\tikzset{external/force remake}
%\tikzexternalize[shell escape=-enable-write18]

% Configurações do CircuiTiKz
\ctikzset{bipoles/thickness=1}
%\ctikzset{bipoles/length=1.2cm}
\ctikzset{monopoles/ground/width/.initial=.2}
\ctikzset{bipoles/resistor/height=0.25}
\ctikzset{bipoles/resistor/width=0.6}
\ctikzset{bipoles/capacitor/height=0.5}
\ctikzset{bipoles/capacitor/width=0.15}
\ctikzset{bipoles/generic/height=0.25}
\ctikzset{bipoles/generic/width=0.6}
%\ctikzset{bipoles/capacitor polar/length=0.5}
%\ctikzset{bipoles/diode/height=.375}
%\ctikzset{bipoles/diode/width=.3}
%\ctikzset{tripoles/thyristor/height=.8}
%\ctikzset{tripoles/thyristor/width=1}
\ctikzset{bipoles/vsourcesin/height=.5}
\ctikzset{bipoles/vsourcesin/width=.5}
\ctikzset{bipoles/cvsourceam/height=.6}
\ctikzset{bipoles/cvsourceam/width=.6}
%\ctikzset{tripoles/european controlled voltage source/width=.4}

\tikzstyle{every node}=[font=\footnotesize]
\tikzstyle{every path}=[line width=0.25pt,line cap=round,line join=round]
%\tikzstyle{every path}=[line cap=round,line join=round]


% Definição de cores MATLAB
\definecolor{matlab_blue}{rgb}	{         0,    0.4470,    0.7410}
\definecolor{matlab_orange}{rgb}{    0.8500,    0.3250,    0.0980}
\definecolor{matlab_yellow}{rgb}{    0.9290,    0.6940,    0.1250}
\definecolor{matlab_violet}{rgb}{    0.4940,    0.1840,    0.5560}
\definecolor{matlab_green}{rgb}	{	 0.4660,    0.6740,    0.1880}
\definecolor{matlab_lblue}{rgb}	{    0.3010,    0.7450,    0.9330}
\definecolor{matlab_red}{rgb}	{    0.6350,    0.0780,    0.1840}

% Personalização das legendas
\usepackage[format = plain, %hang
			justification = centering,
			labelsep = endash,
			singlelinecheck = false,
			skip = 6pt,
			listformat = simple]{caption}	

% Personalização das unidades
\sisetup{output-decimal-marker = {,}}
\sisetup{exponent-product = \cdot}
\sisetup{tight-spacing=true}
\sisetup{group-digits = false}

% Personalizações de tipo de colunas de tabelas
\newcolumntype{L}[1]{>{\raggedright\let\newline\\\arraybackslash\hspace{0pt}}m{#1}}
\newcolumntype{C}[1]{>{\centering\let\newline\\\arraybackslash\hspace{0pt}}m{#1}}
\newcolumntype{R}[1]{>{\raggedleft\let\newline\\\arraybackslash\hspace{0pt}}m{#1}}

% CONFIGURAÇÕES DE PACOTES
% Configurações do pacote backref
% Usado sem a opção hyperpageref de backref
\renewcommand{\backrefpagesname}{Citado na(s) página(s):~}
% Texto padrão antes do número das páginas
\renewcommand{\backref}{}
% Define os textos da citação
\renewcommand*{\backrefalt}[4]{
	\ifcase #1 %
	Nenhuma citação no texto.%
	\or
	Citado na página #2.%
	\else
	Citado #1 vezes nas páginas #2.%
	\fi}%

% alterando o aspecto da cor azul
%\definecolor{blue}{RGB}{41,5,195}

% informações do PDF
\makeatletter
\hypersetup{
	%pagebackref=true,
	pdftitle={\@title}, 
	pdfauthor={\@author},
	pdfsubject={\imprimirpreambulo},
	pdfcreator={LaTeX with abnTeX2},
	pdfkeywords={abnt}{latex}{abntex}{abntex2}{trabalho academico}, 
	colorlinks=true,       		% false: boxed links; true: colored links
	linkcolor=black,          	% color of internal links
	citecolor=black,        	% color of links to bibliography
	filecolor=black,      		% color of file links
	urlcolor=black,
	bookmarksdepth=4
}
\makeatother


\makeatletter
\newcommand{\includetikz}[1]{%
	\tikzsetnextfilename{#1}%
	\input{#1.tex}%
}
\makeatother


% ---
% Possibilita criação de Quadros e Lista de quadros.
% Ver https://github.com/abntex/abntex2/issues/176
%
\newcommand{\quadroname}{Quadro}
\newcommand{\listofquadrosname}{Lista de quadros}

\newfloat[chapter]{quadro}{loq}{\quadroname}
\newlistof{listofquadros}{loq}{\listofquadrosname}
\newlistentry{quadro}{loq}{0}

% configurações para atender às regras da ABNT
\setfloatadjustment{quadro}{\centering}
\counterwithout{quadro}{chapter}
\renewcommand{\cftquadroname}{\quadroname\space} 
\renewcommand*{\cftquadroaftersnum}{\hfill--\hfill}

\setfloatlocations{quadro}{hbtp} % Ver https://github.com/abntex/abntex2/issues/176
% ---


% Espaçamento depois do título
\setlength{\afterchapskip}{0.7\baselineskip}
% O tamanho do parágrafo é dado por:
\setlength{\parindent}{1.25cm}
% Controle do espaçamento entre um parágrafo e outro:
\setlength{\parskip}{0.0cm}  % tente também \onelineskip
%\SingleSpacing % Espaçamento simples 
\OnehalfSpacing % Espaçamento 1,5 (UDESC/CCT)
%\DoubleSpacing	% Espaçamento duplo

% ---
% Margens - NBR 14724/2011 - 5.1 Formato
% ---
\setlrmarginsandblock{3cm}{2cm}{*}
\setulmarginsandblock{3cm}{2cm}{*}
\checkandfixthelayout[fixed]
% ---


% To use externalize consider
%https://tex.stackexchange.com/questions/182783/tikzexternalize-not-compatible-with-miktex-2-9-abntex2-package
%Lauro Cesar digged into the problem until he came with a solution for me to test. And it Works!
%
%According to this link:
%
%The package calc changed the commands \setcounter and friends to be fragile. So you have to make them robust. The example below uses etoolbox with \robustify:
%
\usepackage{etoolbox}
\robustify\setcounter
\robustify\addtocounter
\robustify\setlength
\robustify\addtolength


%% How to silence memoir class warning against the use of caption package?
%% https://tex.stackexchange.com/questions/391993/how-to-silence-memoir-class-warning-against-the-use-of-caption-package
%\usepackage{silence}
%\WarningFilter*{memoir}{You are using the caption package with the memoir class}
%\WarningFilter*{Class memoir Warning}{You are using the caption package with the memoir class}

% --------------------------------------------------------
% INICIO DAS CUSTOMIZACOES PARA A UDESC
% --------------------------------------------------------

% --------------------------------------------------------
% Fontes padroes de part, chapter, section, subsection e subsubsection
% --------------------------------------------------------
% --- Chapter ---
\renewcommand{\ABNTEXchapterfont}{\fontseries{b}} %\bfseries
\renewcommand{\ABNTEXchapterfontsize}{\normalsize}
% --- Part ---
\renewcommand{\ABNTEXpartfont}{\ABNTEXchapterfont}
\renewcommand{\ABNTEXpartfontsize}{\LARGE}
% --- Section ---
\renewcommand{\ABNTEXsectionfont}{\normalfont}
\renewcommand{\ABNTEXsectionfontsize}{\normalsize}
% --- SubSection ---
\renewcommand{\ABNTEXsubsectionfont}{\fontseries{b}} %\bfseries
\renewcommand{\ABNTEXsubsectionfontsize}{\normalsize}
% --- SubSubSection ---
\renewcommand{\ABNTEXsubsubsectionfont}{\itshape}
\renewcommand{\ABNTEXsubsubsectionfontsize}{\normalsize}

\renewcommand{\ABNTEXsubsubsubsectionfont}{\normalfont}
\renewcommand{\ABNTEXsubsubsubsectionfontsize}{\normalsize}
% ---

% --------------------------------------------------------
% Fontes das entradas do sumario
% --------------------------------------------------------

\renewcommand{\cftpartfont}{\ABNTEXpartfont\selectfont}
\renewcommand{\cftpartpagefont}{\normalsize\selectfont}

\renewcommand{\cftchapterfont}{\ABNTEXchapterfont\selectfont}
\renewcommand{\cftchapterpagefont}{\normalsize\selectfont}

\renewcommand{\cftsectionfont}{\ABNTEXsectionfont\selectfont}
\renewcommand{\cftsectionpagefont}{\normalsize\selectfont}

\renewcommand{\cftsubsectionfont}{\ABNTEXsubsectionfont\selectfont}
\renewcommand{\cftsubsectionpagefont}{\normalsize\selectfont}

\renewcommand{\cftsubsubsectionfont}{\normalfont\itshape\selectfont}
\renewcommand{\cftsubsubsectionpagefont}{\normalsize\selectfont}

\renewcommand{\cftparagraphfont}{\normalfont\selectfont}
\renewcommand{\cftparagraphpagefont}{\normalsize\selectfont}

% --------------------------------------------------------
% Usando os pacotes hyperref, uppercase... 
% Para deixar a section do toc uppercase precisa de:
% --------------------------------------------------------
\usepackage{textcase}

\makeatletter

\let\oldcontentsline\contentsline
\def\contentsline#1#2{%
	\expandafter\ifx\csname l@#1\endcsname\l@section
	\expandafter\@firstoftwo
	\else
	\expandafter\@secondoftwo
	\fi
	{%
		\oldcontentsline{#1}{\MakeTextUppercase{#2}}%
	}{%
		\oldcontentsline{#1}{#2}%
	}%
}
\makeatother

% --------------------------------------------------------
% Renomenando as entradas de APÊNDICES E ANEXOS
% --------------------------------------------------------

\renewcommand{\apendicesname}{AP\^ENDICES}
\renewcommand{\anexosname}{ANEXOS}


% Manipulação de Strings
%\RequirePackage{xstring}

% Comando para inverter sobrenome e nome
\newcommand{\invertname}[1]{%
	\StrBehind{#1}{{}}, \StrBefore{#1}{{}}%
}%


% --------------------------------------------------------
% Alterando os estilos de Caption e Fonte
% --------------------------------------------------------
\makeatletter
% Define o comando \fonte que respeita as configurações de caption do memoir ou do caption
\renewcommand{\fonte}[2][\fontename]{%
	\M@gettitle{#2}%
	\memlegendinfo{#2}%
	\par
	\begingroup
	\@parboxrestore
	\if@minipage
	\@setminipage
	\fi
	\ABNTEXfontereduzida
	\configureseparator
	\captiondelim{\ABNTEXcaptionfontedelim}
	\@makecaption{#1}{\ignorespaces #2}\par
	\endgroup}


\captionstyle[\raggedright]{\raggedright}

\makeatother

\setlength{\cftbeforechapterskip}{0pt plus 0pt}
\renewcommand*{\insertchapterspace}{}

\newlength{\mylen}	% New length to use with spacing
\setlength{\mylen}{1pt}

\setlength{\cftbeforechapterskip}{\mylen}
\setlength{\cftbeforesectionskip}{\mylen}
\setlength{\cftbeforesubsectionskip}{\mylen}
\setlength{\cftbeforesubsubsectionskip}{\mylen}
\setlength{\cftbeforesubsubsubsectionskip}{\mylen}


% ---
% Ajuste das listas de abreviaturas e siglas ; e símbolos [Personalizada para UDESC com espaçamento 1,5]
% ---

% ---
% Redefinição da Lista de abreviaturas e siglas [Personalizada para UDESC com espaçamento 1,5]
\renewenvironment{siglas}{%
	\pretextualchapter{\listadesiglasname}
	\begin{symbols} 
		\setlength{\itemsep}{0pt}	% Ajuste para Espaçamento 1,5 (UDESC/CCT)
	}{% 
	\end{symbols}
	\cleardoublepage
}
% ---

% ---
% Redefinição da Lista de símbolos [Personalizada para UDESC com espaçamento 1,5]
\renewenvironment{simbolos}{%
	\pretextualchapter{\listadesimbolosname}
	\begin{symbols}
		\setlength{\itemsep}{0pt}	% Ajuste para Espaçamento 1,5 (UDESC/CCT)
	}{%
	\end{symbols}
	\cleardoublepage
}
% ---


% ---
% Remocao dos simbolos de < > das urls, ver manual pacote url pg 6 item 6
% https://github.com/abntex/biblatex-abnt/issues/16
\def\UrlLeft{}
\def\UrlRight{}
% ---

% ---
% FIM DAS CUSTOMIZACOES PARA A  Universidade do Estado de Santa Catarina - UDESC/CCT
% ---







% -----------------------------------------------------------------
% Informações de dados para CAPA e FOLHA DE ROSTO
% -----------------------------------------------------------------
\titulo{Trabalho Prático com as Ferramentas Zabbix e Grafana}%

\autor{Gustavo Henrique {}Trassi Ganaza \and Yoshiyuki {}Fugie}
\orientador{Sherlock Holmes {}da Silva}
%\coorientador{Arnold Alois {}Schwarzenegger}

\instituicao{Universidade Estadual de Maringá, Centro de Tecnologia, Graduação em Ciência da Computação}%

\tipotrabalho{Trabalho de Disciplina}

\preambulo{Monografia de trabalho de conclusão de curso apresentada apresentada ao Curso de Ciência da Computação, Centro de Tecnologia, Departamento de Informática da Universidade Estadual de Maringá, como parte dos requisitos necessários para a obtenção do grau de Bacharel em Ciência da Computação.}

\local{Maringá}%

\data{\the\year}%
% ---

% compila o indice
\makeindex

% -----------------------------------------------------------------
% Início do documento
% -----------------------------------------------------------------
\begin{document}

\selectlanguage{brazil}
\frenchspacing  % Retira espaço extra obsoleto entre as frases.

% -----------------------------------------------------------------
% ELEMENTOS PRÉ-TEXTUAIS
% -----------------------------------------------------------------
\pretextual

% Você pode comentar os elementos que não deseja em seu trabalho;

% A capa pode ser escolhida dentro do arquivo Capa.tex (TCC, Master, Doc, ...)
% ---
% Capa
% ---


% --------------------------------------------------------
% Capa Padrão
% --------------------------------------------------------
\renewcommand{\imprimircapa}{%
	\begin{capa}%
		\center

		{\fontseries{b}\selectfont\MakeTextUppercase{UNIVERSIDADE ESTADUAL DE MARINGÁ -- UEM}}
		
		{\fontseries{b}\selectfont\MakeTextUppercase{CENTRO DE TECNOLOGIA -- CTC}}

        {\fontseries{b}\selectfont\MakeTextUppercase{DEPARTAMENTO DE INFORMÁTICA -- DIN}}
		
		{\fontseries{b}\selectfont\MakeTextUppercase{BACHARELADO -- CIÊNCIA DA COMPUTAÇÃO  }}
		
		\vfill
		
		{\fontseries{b}\selectfont\MakeTextUppercase{\normalsize\imprimirautor}}
		
		\vfill
		\begin{center}
			{\fontseries{b}\selectfont\MakeTextUppercase{\imprimirtitulo}}
		\end{center}
		\vfill
		
		\vfill
		
		{\fontseries{b}\selectfont\MakeTextUppercase{\imprimirlocal}}
		\par
		{\fontseries{b}\selectfont \imprimirdata}
		\vspace*{1cm}
	\end{capa}
}

\imprimircapa				% Capa padrão					% Elemento Obrigatório
%z% ---
% Folha de rosto
% ---








% --------------------------------------------------------
% folha de rosto 
% --------------------------------------------------------

\makeatletter

\renewcommand{\folhaderostocontent}{
	\begin{center}
		
		{\fontseries{b}\selectfont\MakeTextUppercase{\imprimirautor}}
		
		\vfill
		
		\begin{center}
			{\fontseries{b}\selectfont\MakeTextUppercase{\imprimirtitulo}}
		\end{center}
	
		\vspace*{1.5cm}

		\abntex@ifnotempty{\imprimirpreambulo}{%
			\hspace{.45\textwidth}
			{\begin{minipage}{.5\textwidth}
					\SingleSpacing
					\imprimirpreambulo\par
					\vspace*{4pt}
					{\imprimirorientadorRotulo~\imprimirorientador\par}
					\abntex@ifnotempty{\imprimircoorientador}{%
						{\imprimircoorientadorRotulo~\imprimircoorientador}%
					}%
			\end{minipage}}%
		}%
	
		
		\vfill
		
	{\fontseries{b}\selectfont\MakeTextUppercase{\imprimirlocal}}
	\par
	{\fontseries{b}\selectfont \imprimirdata}
	\vspace*{1cm}
	\end{center}
}


% (o * indica que haverá a ficha bibliográfica)
% ---
\imprimirfolhaderosto*
% ---


			% Elemento Obrigatório
% Caso não utilize a Ficha Catalográfica entre na folha de rosto e retire o * de dentro do arquivo FolhadeRosto
%
% ---
% Inserir a ficha bibliografica
% ---

% Isto é um exemplo de Ficha Catalográfica, ou ``Dados internacionais de
% catalogação-na-publicação''. Você pode utilizar este modelo como referência. 
% Porém, provavelmente a biblioteca da sua universidade lhe fornecerá um PDF
% com a ficha catalográfica definitiva após a defesa do trabalho. Quando estiver
% com o documento, salve-o como PDF no diretório do seu projeto e substitua todo
% o conteúdo de implementação deste arquivo pelo comando abaixo:



% \begin{fichacatalografica}
%     \includepdf{fig_ficha_catalografica.pdf}
% \end{fichacatalografica}


%	\setlength{\parindent}{0cm}
%	\setlength{\parskip}{0pt}
\begin{fichacatalografica}
	%\sffamily
	%\rmfamily
	\ttfamily \hbadness=10000
	\vspace*{\fill}					% Posição vertical
	\begin{center}					% Minipage Centralizado
	
	\vspace*{8pt}
	
%	\begin{minipage}[c]{8cm}
%	\centering \sffamily
%	 Ficha catalográfica elaborada pelo(a) autor(a), com auxílio do programa de geração automática da Biblioteca Setorial do CCT/UDESC
%	\end{minipage}
	\fbox{\begin{minipage}[c]{12.5cm}		% Largura
	\flushright
	{\begin{minipage}[c]{10.5cm}		% Largura
	\vspace{1.25cm}
	%\footnotesize
	\setlength{\parindent}{1.5em}
	\noindent \invertname{\imprimirautor} \par
	\imprimirtitulo{ }/{ }\imprimirautor. -- \imprimirlocal, \imprimirdata .\par
	\pageref{LastPage} p. : il. ; 30 cm.\par
	\vspace{1.5em}
	\imprimirorientadorRotulo~\imprimirorientador.\par
	\imprimircoorientadorRotulo~\imprimircoorientador.\par
	\imprimirtipotrabalho~--~\imprimirinstituicao, \imprimirlocal, \imprimirdata.\par
	\vspace{1.5em}
		1. Palavra-chave.
		2. Palavra-chave.
		3. Palavra-chave.
 		4. Palavra-chave.
		5. Palavra-chave.
		I. \invertname{\imprimirorientador}.
		II. \invertname{\imprimircoorientador}.
		III. \imprimirinstituicao.
		IV. Título. %
	\vspace{1.25cm}	%		
	\end{minipage}%
	}% 
	\end{minipage}}%
	
	\vspace*{0.5cm}
	
	\end{center}
\end{fichacatalografica}


%\begin{fichacatalografica}
%	\sffamily
%	\vspace*{\fill}					% Posição vertical
%	\begin{center}					% Minipage Centralizado
%	\fbox{\begin{minipage}[c][8cm]{13.5cm}		% Largura
%	\small
%	\imprimirautor
%	%Sobrenome, Nome do autor
%	
%	\hspace{0.5cm} \imprimirtitulo  / \imprimirautor. --
%	\imprimirlocal, \imprimirdata-
%	
%	\hspace{0.5cm} \pageref{LastPage} p. : il. (algumas color.) ; 30 cm.\\
%	
%	\hspace{0.5cm} \imprimirorientadorRotulo~\imprimirorientador\\
%	
%	\hspace{0.5cm}
%	\parbox[t]{\textwidth}{\imprimirtipotrabalho~--~\imprimirinstituicao,
%	\imprimirdata.}\\
%	
%	\hspace{0.5cm}
%		1. Palavra-chave1.
%		2. Palavra-chave2.
%		3. Palavra-chave3.
% 		4. Palavra-chave4.
%		5. Palavra-chave5.
%		I. Orientador.
%		II. Universidade xxx.
%		III. Faculdade de xxx.
%		IV. Título 			
%	\end{minipage}}
%	\end{center}
%\end{fichacatalografica}
% ---

	% Elemento Obrigatório (Verso da Folha)
%
% ---
% Inserir errata
% ---
\begin{errata}
Elemento opcional. 

Exemplo:

\vspace{\onelineskip}

SOBRENOME, Prenome do Autor. Título de obra: subtítulo (se houver). Ano de depósito. Tipo do trabalho (grau e curso) - Vinculação acadêmica, local de apresentação/defesa, data.

\begin{table}[htb]
\center
\begin{tabular}{|p{2.4cm}|p{2cm}|p{3cm}|p{3cm}|}
  \hline
   \textbf{Folha} & \textbf{Linha}  & \textbf{Onde se lê}  & \textbf{Leia-se}  \\
    \hline
    1 & 10 & auto-conclavo & autoconclavo\\
   \hline
\end{tabular}
\end{table}

\end{errata}
% ---				% Elemento Opcional
%
% ---
% Inserir folha de aprovação
% ---

% Isto é um exemplo de Folha de aprovação, elemento obrigatório da NBR
% 14724/2011 (seção 4.2.1.3). Você pode utilizar este modelo até a aprovação
% do trabalho. Após isso, substitua todo o conteúdo deste arquivo por uma
% imagem da página assinada pela banca com o comando abaixo:
%
% \includepdf{folhadeaprovacao_final.pdf}
%
\begin{folhadeaprovacao}



	\begin{center}
		{\fontseries{b}\selectfont\MakeTextUppercase{\normalsize\imprimirautor}}
	\end{center}
    \vfill
    
	\vfill
	\begin{center}
		{\fontseries{b}\selectfont\MakeTextUppercase{\imprimirtitulo}}
	\end{center}
	\vfill

    
\abntex@ifnotempty{\imprimirpreambulo}{%
	\hspace{.45\textwidth}
	{\begin{minipage}{.5\textwidth}
			\SingleSpacing
			\imprimirpreambulo\par
			\vspace*{4pt}
			{\imprimirorientadorRotulo~\imprimirorientador\par}
			\abntex@ifnotempty{\imprimircoorientador}{%
				{\imprimircoorientadorRotulo~\imprimircoorientador}%
			}%
	\end{minipage}}%
}%


\vfill
        
	 \begin{center}
	 	
    	{\fontseries{b}\selectfont BANCA EXAMINADORA: }
    	\vspace*{1.75cm}
    
		Nome do Orientador e Titulação \par
		Nome da Instituição
	 \end{center}
	
    {Membros:} 
    
	\begin{center}
		\vspace*{1.25cm}
		Nome do Orientador e Titulação \par
		Nome da Instituição
		
		\vspace*{1.25cm}
		Nome do Orientador e Titulação \par
		Nome da Instituição
		
		\vspace*{1.25cm}
		Nome do Orientador e Titulação \par
		Nome da Instituição

	
	\end{center}
    
    \vspace*{\fill}  
    \begin{center}
    {\imprimirlocal, 01 de maio de \imprimirdata}
	\end{center}
    \vspace*{0.25cm}  
\end{folhadeaprovacao}
% ---




%\textbf{	{Orientador: \vspace{-16pt} }
%	\assinatura{\textbf{Prof. \imprimirorientador , Dr.} \\ Univ. XXX} 
%	{Coorientador: \vspace{-16pt}}   
%	\assinatura{\textbf{Prof. \imprimircoorientador , Dr.} \\ Univ. XXX}
%	
%	{Membros: \vspace{-16pt} } 
%	
%	% --- Exemplo de assinaturas em sequência ---       
%	\setlength{\ABNTEXsignwidth}{8.5cm}
%	
%	\assinatura{\textbf{Prof. Professor, Dr.} \\ Univ. XXX}
%	\assinatura{\textbf{Prof. Professor, Dr.} \\ Univ. XXX}
%	\assinatura{\textbf{Prof. Professor, Dr.} \\ Univ. XXX}
%	
%	% --- Exemplo de assinaturas lado a lado ---
%	\setlength{\ABNTEXsignwidth}{7.5cm}
	%
	%    \noindent\hfill\assinatura*{\textbf{Prof. Professor, Dr.} \\ Univ. XXX}%
	%    \hfill%
	%    \assinatura*{\textbf{Prof. Professor, Dr.} \\ Univ. XXX}%
	%    \hfill
	%    
	%    \noindent\hfill\assinatura*{\textbf{Prof. Professor, Dr.} \\ Univ. XXX}%
	%    \hfill%
	%    \assinatura*{\textbf{Prof. Professor, Dr.} \\ Univ. XXX}%
	%    \hfill}		% Elemento Obrigatório
%% ---
% Dedicatória
% ---
\begin{dedicatoria}
   \vspace*{\fill}
%   \begin{flushright}
%   \noindent
%	Este trabalho é dedicado às crianças adultas que,\\
%	quando pequenas, sonharam em se tornar cientistas. 
%   \end{flushright}

{%
	\noindent\hspace{.5\textwidth}
	{\begin{minipage}{.5\textwidth}
			\begin{flushleft}
				Aos torcedores corinthianos, pela inspiração de sempre!
			\end{flushleft}
	\end{minipage}}%
\vspace*{3cm}
}%

\end{dedicatoria}
% ---
			% Elemento Opcional
%% ---
% Agradecimentos
% ---
\begin{agradecimentos}
Agradeço ao meu orientador por aceitar conduzir o meu trabalho de pesquisa.
A todos os torcedores da gigante torcida corinthiana pela excelência da qualidade técnica de cada um.

Aos meus pais que sempre estiveram ao meu lado me apoiando ao longo de toda a minha trajetória. Sou grato à minha família pelo apoio que sempre me deram durante toda a minha vida.

Como disse Snoop Dog: ``Eu quero me agradecer por acreditar em mim mesmo, quero me agradecer por todo esse trabalho duro. Quero me agradecer por não tirar folgas. Quero me agradecer por nunca desistir. Quero me agradecer por ser generoso e sempre dar mais do que recebo. Quero me agradecer por tentar sempre fazer mais o certo do que o errado. Quero me agradecer por ser eu mesmo o tempo inteiro''.

Deixo um agradecimento especial ao meu orientador pelo incentivo e pela dedicação do seu escasso tempo ao meu projeto de pesquisa.


\end{agradecimentos}
% ---		% Elemento Opcional
%% ---
% Epígrafe
% ---
\begin{epigrafe}
    \vspace*{\fill}
%	\begin{flushright}
%		\textit{``Eu não falhei, encontrei 10 mil soluções que não davam certo.'' (EDISON, [19--])}
%	\end{flushright}
{%
	\noindent\hspace{.5\textwidth}
	{\begin{minipage}{.5\textwidth}
		\begin{flushright}
			``Eu não falhei, encontrei 10 mil soluções que não davam certo.'' (EDISON, [19--])
		\end{flushright}
	\end{minipage}}%
	\vspace*{3cm}
}%
\end{epigrafe}
% ---				% Elemento Opcional
% ---
% RESUMOS
% ---

% resumo em português
\setlength{\absparsep}{18pt} % ajusta o espaçamento dos parágrafos do resumo
\begin{resumo}
Este relatório detalha o processo de implementação de uma solução de monitoramento de infraestrutura de TI utilizando as ferramentas de código aberto Zabbix e Grafana. O objetivo principal do trabalho é a instalação, configuração e utilização conjunta dessas plataformas para monitorar ativamente um dispositivo de rede, especificamente uma máquina virtual executando o sistema operacional Linux. O escopo do projeto abrange desde a coleta de métricas essenciais de desempenho, como uso de CPU, memória e disco, até a configuração de mecanismos de alerta proativos. Foram criadas triggers personalizadas para a detecção de anomalias, incluindo a verificação de integridade de arquivos críticos do sistema, como o arquivo de senhas do servidor. Adicionalmente, o sistema foi configurado para notificar os administradores via e-mail em caso de ativação desses gatilhos. A integração com a ferramenta Grafana permitiu a criação de dashboards visualmente ricos e intuitivos, facilitando a análise e a interpretação dos dados coletados. Ao final, este trabalho demonstra a eficácia e a flexibilidade da combinação Zabbix-Grafana como uma solução robusta e de baixo custo para o gerenciamento de redes.

 \textbf{Palavras-chave}: Zabbix. Grafana. Monitoramento. Infraestrutura de TI. Análise de Dados.
\end{resumo}
				% Elemento Obrigatório
%% ---
% Abstract
% ---

% resumo em inglês
\begin{resumo}[Abstract]
 \begin{otherlanguage*}{english}
   Elemento obrigatório para todos os trabalhos de conclusão de curso. Opcional para os demais trabalhos acadêmicos, inclusive para artigo científico. Constitui a versão do resumo em português para um idioma de divulgação internacional. Deve aparecer em página distinta e seguindo a mesma formatação do resumo em português.

   \textbf{Keywords}: Keyword 1. Keyword 2. Keyword 3. Keyword 4. Keyword 5.
 \end{otherlanguage*}
\end{resumo}
				% Elemento Obrigatório
%
% ---
% inserir lista de ilustrações
% ---
\pdfbookmark[0]{\listfigurename}{lof}
\listoffigures*
\cleardoublepage
% ---

% ---
% inserir lista de quadros
% ---
%\pdfbookmark[0]{\listofquadrosname}{loq}
%\listofquadros*
%\cleardoublepage
% ---


% ---
% inserir lista de tabelas
% ---
\pdfbookmark[0]{\listtablename}{lot}
\listoftables*
\cleardoublepage
% ---

% ---
% inserir lista de abreviaturas e siglas
% ---
\begin{siglas}
	\item[ABNT] Associação Brasileira de Normas Técnicas
	\item[BU] Biblioteca Universitária
	\item[IN] Instrução Normativa
	\item[NBR] Normas Técnicas Brasileiras
	\item[TCC] Trabalho de Conclusão de Curso
	\item[UEM] Universidade Estadual de Maringá
\end{siglas}
% ---

% ---
% inserir lista de símbolos
% ---


\begin{simbolos}
  \item[@] Arroba
  \item[\%] Porcento
  \item[$^\circ$C] Graus Celsius
  \item[Ca] Cálcio
\end{simbolos}

% ---
				% Elemento Opcional
%% ---
% inserir o sumario
% ---
\pdfbookmark[0]{\contentsname}{toc}
\tableofcontents*
\cleardoublepage
% ---
				% Elemento Obrigatório

% -----------------------------------------------------------------
% ELEMENTOS TEXTUAIS
% -----------------------------------------------------------------
\textual

\pagestyle{PagNumReduzida}						% Comando para cabeçalho somente com numeração de página 10pt
\aliaspagestyle{chapter}{PagNumReduzida}		% Deixar numeração da primeira página com tamanho igual ao resto da numeração
% ref.: https://groups.google.com/g/abntex2/c/CP7g8ZMgi-c/m/KjfEnn5b9a4J


% ---- Mantenha está estrutura, assim você deixa o trabalho mais organizado -------

\chapter{Introdução}
A importância do gerenciamento reside na sua capacidade de fornecer visibilidade e controle sobre todos os componentes da rede, desde servidores e estações de trabalho até roteadores e switches. Um gerenciamento eficaz permite a identificação proativa de problemas, antes que eles impactem os usuários finais. Por meio do monitoramento contínuo de métricas de desempenho, como latência, utilização de banda e carga de processamento, os administradores podem prever gargalos, otimizar a alocação de recursos e planejar expansões de forma mais assertiva.

Além disso, em um contexto de ameaças cibernéticas crescentes, o gerenciamento de redes desempenha um papel crucial na segurança. O monitoramento de padrões de tráfego, logs de sistema e integridade de arquivos permite a detecção de atividades suspeitas ou não autorizadas, servindo como uma camada essencial de defesa. Diante dessa complexidade, o uso de ferramentas especializadas torna-se indispensável. Soluções como o Zabbix e o Grafana capacitam as equipes de TI a automatizar a coleta de dados, analisar tendências e responder rapidamente a incidentes, transformando uma tarefa complexa em um processo sistemático e controlado. Este trabalho explora a aplicação prática dessas ferramentas para construir uma solução de monitoramento completa.

\chapter{Descrição das Ferramentas}
Para a realização deste trabalho, foram utilizadas duas das mais populares ferramentas de código aberto no ecossistema de monitoramento e observabilidade: Zabbix e Grafana. Embora ambas atuem no mesmo domínio, elas possuem funções distintas e complementares que, quando integradas, formam uma solução extremamente poderosa.

\section{Zabbix}
O Zabbix é uma solução de monitoramento de nível empresarial, robusta e escalável, projetada para monitorar a disponibilidade e o desempenho de toda a infraestrutura de TI. Ele é capaz de coletar métricas de praticamente qualquer tipo de dispositivo, incluindo servidores, máquinas virtuais, aplicações, bancos de dados e equipamentos de rede. As principais características do Zabbix incluem:

\begin{itemize}
    \item \textbf{Coleta de Dados Flexível:} O Zabbix utiliza múltiplos métodos para coletar dados, sendo os principais o Zabbix Agent (um software leve instalado nos hosts monitorados) e métodos \textit{agentless}, como SNMP, ICMP e verificações simples de serviços (HTTP, FTP, etc.).
    \item \textbf{Detecção de Problemas (Triggers):} Sua principal força reside na capacidade de definir expressões lógicas, chamadas de \textit{triggers}, que analisam os dados coletados em tempo real. As triggers possuem níveis de severidade customizáveis e são ativadas quando uma condição anômala é detectada (ex: uso de CPU acima de 90\%).
    \item \textbf{Sistema de Alertas e Ações:} Uma vez que uma trigger é ativada, o Zabbix pode executar ações pré-configuradas, como enviar notificações por e-mail, SMS ou outras plataformas de mensagens, além de executar scripts remotos para tentar solucionar o problema automaticamente.
    \item \textbf{Visualização Nativa:} Possui uma interface web completa que permite a criação de dashboards, mapas de rede e gráficos para visualizar os dados coletados e o status dos problemas.
\end{itemize}

\section{Grafana}
O Grafana é uma plataforma de código aberto especializada em visualização e análise de dados. Diferente do Zabbix, seu foco principal não é a coleta de dados, mas sim a capacidade de se conectar a diversas fontes de dados (\textit{data sources}) para criar dashboards interativos, unificados e esteticamente avançados. As principais características do Grafana incluem:

\begin{itemize}
    \item \textbf{Múltiplas Fontes de Dados:} O Grafana se destaca por sua habilidade de se conectar a dezenas de fontes de dados simultaneamente, como bancos de dados (Prometheus, InfluxDB, MySQL), serviços em nuvem e, para o propósito deste trabalho, o Zabbix.
    \item \textbf{Dashboards Ricos e Interativos:} Permite a criação de painéis altamente customizáveis, com uma vasta gama de tipos de visualização (gráficos de linha, barras, medidores, mapas de calor, etc.). Os usuários podem explorar os dados de forma interativa, aplicando zoom, filtros e mudando o intervalo de tempo dinamicamente.
    \item \textbf{Sistema de Plugins:} Sua arquitetura extensível, baseada em plugins, permite que a comunidade desenvolva novas integrações de fontes de dados e novos painéis de visualização, tornando a ferramenta extremamente versátil.
    \item \textbf{Alertas Visuais:} Assim como o Zabbix, o Grafana também possui um sistema de alertas que pode ser configurado diretamente a partir dos gráficos nos dashboards, notificando os usuários quando uma métrica ultrapassa um determinado limiar.
\end{itemize}			% Capítulo 1
% -----------------------------------------------------------------
% ELEMENTOS PÓS-TEXTUAIS
% -----------------------------------------------------------------
\postextual

% Você pode comentar os elementos que não deseja em seu trabalho;

% Referências bibliográficas

%Notar que os autores continuam sendo transpostos em maiúsculas, como preconiza a ABNT NBR 6023:2002. 
%Se, no entanto, não desejar seguir esta regra,
%crie um novo arquivo .bst (por exemplo, novoestilo.bst)a partir do estilo
%usado (abntex2-cite-alf ou abntex2-cite-num) e retire todas as expressões
%"u" change.case$, lembrando-se de indicar o novo arquivo como estilo, por exemplo, 
%\bibliographystyle{novoestilo}, e colocar o arquivo criado na mesma
%pasta em que está compilando o documento.

% Arquivo alterado para citação (Autor, Ano) ao invés de (AUTOR, Ano) conforme ABNT NBR 10520:2023
\bibliographystyle{citacao-atualizada.bst}	

\bibliography{referencias}	% Elemento Obrigatório

%% ----------------------------------------------------------
% Glossário
% ----------------------------------------------------------

%Consulte o manual da classe abntex2 para orientações sobre o glossário.

%\glossary




% ----------------------------------------------------------
% Glossário (Formatado Manualmente)
% ----------------------------------------------------------

\chapter*{GLOSSÁRIO}
\addcontentsline{toc}{chapter}{GLOSSÁRIO}

{ \setlength{\parindent}{0pt} % ambiente sem indentação

\textbf{Ardósia}: Rocha metamórfica sílico-argilosa formada pela transformação da argila sob pressão e temperatura, endurecida em finas lamelas.

\textbf{Arenito}: rocha sedimentária de origem detrítica formada de grãos agregados por um cimento natural silicoso, calcário ou ferruginoso que comunica ao conjunto em geral qualidades de dureza e compactação.

\textbf{Feldspato}: grupo de silicatos de sódio, potássio, cálcio ou outros elementos que compreende dois subgrupos, os feldspatos alcalinos e os plagioclásios.






} % fim ambiente sem indentação


				% Elemento Opcional
%
% ----------------------------------------------------------
% Apêndices
% ----------------------------------------------------------

% ---
% Inicia os apêndices
% ---
\begin{apendicesenv}

% Imprime uma página indicando o início dos apêndices
%\partapendices

% ----------------------------------------------------------
\chapter{TÍTULO}
% ----------------------------------------------------------


\end{apendicesenv}
% ---				% Elemento Opcional
%
% ----------------------------------------------------------
% Anexos
% ----------------------------------------------------------
%
% ---
% Inicia os anexos
% ---
\begin{anexosenv}

% Imprime uma página indicando o início dos anexos
%\partanexos

% ---
\chapter{TÍTULO}
% ---



\end{anexosenv}
				% Elemento Opcional
%
%%---------------------------------------------------------------------
%% INDICE REMISSIVO
%%---------------------------------------------------------------------

%\phantompart
%\printindex

%---------------------------------------------------------------------

%%---------------------------------------------------------------------
%% INDICE REMISSIVO (Formatado Manualmente)
%%---------------------------------------------------------------------

\chapter*{ÍNDICE}
\addcontentsline{toc}{chapter}{ÍNDICE}

{ \setlength{\parindent}{0pt}  % ambiente sem indentação
	
Andesito, 22, 50, 73

Argila, 52, 75, 121

Basalto, 25, 230, 235

	
	
	
	
} % fim ambiente sem indentação


		% Elemento Opcional

\end{document}

% -----------------------------------------------------------------
% Fim do Documento
% -----------------------------------------------------------------	