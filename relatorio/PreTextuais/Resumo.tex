% ---
% RESUMOS
% ---

% resumo em português
\setlength{\absparsep}{18pt} % ajusta o espaçamento dos parágrafos do resumo
\begin{resumo}
Este relatório detalha o processo de implementação de uma solução de monitoramento de infraestrutura de TI utilizando as ferramentas de código aberto Zabbix e Grafana. O objetivo principal do trabalho é a instalação, configuração e utilização conjunta dessas plataformas para monitorar ativamente um dispositivo de rede, especificamente uma máquina virtual executando o sistema operacional Linux. O escopo do projeto abrange desde a coleta de métricas essenciais de desempenho, como uso de CPU, memória e disco, até a configuração de mecanismos de alerta proativos. Foram criadas triggers personalizadas para a detecção de anomalias, incluindo a verificação de integridade de arquivos críticos do sistema, como o arquivo de senhas do servidor. Adicionalmente, o sistema foi configurado para notificar os administradores via e-mail em caso de ativação desses gatilhos. A integração com a ferramenta Grafana permitiu a criação de dashboards visualmente ricos e intuitivos, facilitando a análise e a interpretação dos dados coletados. Ao final, este trabalho demonstra a eficácia e a flexibilidade da combinação Zabbix-Grafana como uma solução robusta e de baixo custo para o gerenciamento de redes.

 \textbf{Palavras-chave}: Zabbix. Grafana. Monitoramento. Infraestrutura de TI. Análise de Dados.
\end{resumo}
